 
\section{Introduction}

Causal broadcast (\REF) constitutes the core communication primitive of many
distributed systems (\REF). It ensures the reliable delivery of broadcast
messages, exactly once, following Lamport's happen before
relationship~\cite{lamport1978time}. If the sending of a message $m$ precedes
the sending of a message $m'$ then all processes that deliver these two messages
need to deliver $m$ before $m'$. Otherwise they deliver them in any order.

In large and dynamic systems, processes cannot afford to maintain an up-to-date
knowledge of the full membership. Instead, each process builds a partial view of
neighbors to communicate with. Gossiping exploits these neighborhoods to
efficiently broadcast a message to all processes. To broadcast a message, a
process sends the message to its neighbors; each process receiving such message
forwards it to its own neighbors. Processes receive the message either directly
or transitively. Processes may receive a broadcast message multiples times from
different sources.  To deliver each message exactly once, reliable broadcast
must discard the additional copies of the original message at receipt. 

In static systems where processes cannot join, leave, or self-reconfigure their
neighborhood, the local structure that ensures exactly once delivery is
lightweight.  Since each process forwards each message exactly once, each
process knows that it will receive a number of copies of the original message
equal to the number of their incoming links. Once it received that many copies,
it safely purges its local structure from this message. It will never receive
this message again.

This does not hold in dynamic systems. When a process adds a neighbor to its
partial view, the latter does not know if it should expect another message copy 
from this new incoming link. \TODO{Need to be more accurate.}  To solve this,
causal broadcast relies on reliable broadcast which takes a vector the size of
which is linear with the number of processes that ever broadcast a message
(\REF).

\TODO{We do reliable broadcast on top of causal broadcast. Did not make sense
  before because the cost of causal broadcast was too high. Now that it's cheap,
  we use it to enable reliability at marginal cost.}

% State-of-the-art causal broadcasts (\REF) reuse this vector of clocks. Each
% message piggybacks such vector to ensure their causal delivery. A recent causal
% broadcast (\REF) alleviates processes from the need for piggybacking vectors in
% messages. However, it still requires to maintain the local vector to ensure that
% it delivers each message exactly once.

\begin{table*}
  \begin{center}
    \caption{\label{table:complexity} Complexity of causal broadcast algorithms
      at each process. $N$ the number of processes that ever broadcast a
      message. $W$ the number of messages received but not delivered yet. $B$
      and $D$ are buffers of messages that increase when a link is added but
      purged when the link becomes safe to use. $B$ and $D$ eventually becomes
      empty when the system becomes static. \TODO{Different $N$}}
    \newcommand{\cmark}{\ding{51}}%
\newcommand{\xmark}{\ding{55}}%

\setlength{\tabcolsep}{4pt} % General space between cols (6pt standard)

\small

\begin{tabularx}{0.98\columnwidth}{@{}Xcccc@{}}
  & \makecell{message\\overhead} &  \makecell{delivery\\execution time} & \makecell{local space\\consumption} & \makecell{\# control messages\\per added link} \\ \cmidrule{2-5}
  reliable broadcast~\cite{hadzilacos1994modular} & $O(1)$ & $O(1)$ & $O(N)$ & $0$ \\
  vector-based causal broadcast~\cite{schwarz1994detecting} & $O(N)$ & $O(W.N)$ & $O(N+W.N)$ & $0$ \\ 
  \PCBROADCAST (\REF) & $O(1)$ & $O(1)$ & $O(N)$ & $3$ to $2P^2$ \\ \hline\hline
  \textbf{this paper (\RPCBROADCAST)} & $O(1)$ & $O(Q_i)$ & $\mathbf{O(Q_i.M)}$ & $\mathbf{6}$ to $\mathbf{4P^2}$ \\
%  \bottomrule
\end{tabularx}

%%% Local Variables:
%%% mode: latex
%%% TeX-master: "../paper"
%%% End:

  \end{center}
\end{table*}

In this paper, we present a causal broadcast protocol that extends
(\REF). \TODO{Maybe more generic: on top of any causal broadcast? Along with
  \PCBROADCAST it removes the last linear complexity?} Our contribution is
twofold:
\begin{itemize}
\item We describe a causal broadcast that piggybacks constant size control
  information while maintaining a local structure that is safely purged over
  time, and we prove it.
\item We provide experiments that highlight \TODO{something}.
\end{itemize}
While (\REF) solves the issue about generated traffic, our approach solves the
issue about local space consumption. As consequence, causal broadcast can run in
large and dynamic systems even on most humble devices such as Raspberry Pi's.

\TODO{Not only it's an improvement over causal broadcast but also over reliable
  broadcast if one is okay to introduce a slight delay in messages.}


The rest of this paper is organized as follows. Section~\ref{sec:motivations}
shows the issue and motivates this work. Section~\ref{sec:proposal} presents the
proposed causal broadcast and corresponding
proofs. Section~\ref{sec:experimentation} highlights the pros and cons of our
approach on experimental setups. We conclude in Section~\ref{sec:conclusion}.


%%% Local Variables:
%%% mode: latex
%%% TeX-master: "../paper"
%%% End:
