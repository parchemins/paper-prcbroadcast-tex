
\IEEEtitleabstractindextext{
\begin{abstract}
  Causal broadcast constitutes the fundamental communication primitive of many
  distributed protocols and applications. However, state-of-the-art
  implementations do not scale in large and dynamic systems.  In this paper, we
  propose an original implementation of causal broadcast. Message overhead is
  constant. Local space complexity is non-monotonic and depends on system
  settings and current usage. We prove that all and only obsolete control
  information is discarded, at the cost of few lightweight control messages.
  Our implementation constitutes a better and sustainable communication
  primitive for causal broadcast in large and dynamic systems.
\end{abstract}
}

% \begin{abstract}
%   Causal broadcast constitutes the core communication primitive of many
%   distributed systems. For decades, state-of-the-art approaches relied on
%   maintaining and transmitting vector clocks. The size of vector clocks
%   increases linearly with the number of processes that ever entered the
%   system. Causal broadcast eventually became overcostly and inefficient in large
%   and dynamic systems.  A recent approach solved the issue about generated
%   traffic by removing the need for transmitting vectors. However, it still
%   maintains a vector locally. In this paper, we improve this causal broadcast by
%   removing the need for such vector. The proposed protocol safely purges the
%   local structure over time at cost of few control messages. As consequence,
%   causal broadcast can run in large and dynamic systems even on most humble
%   devices such as Raspberry Pi's.
% \end{abstract}

% \keywords{Causal broadcast, local space complexity}


%%% Local Variables:
%%% mode: latex
%%% TeX-master: "../paper"
%%% End:
