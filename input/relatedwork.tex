
\section{Related work}
\label{sec:relatedwork}

Causal broadcast relies on reliable broadcast to forbid multiple delivery of
messages. This section reviews state-of-the-art from physical clock techniques
to vector-based approaches; from probabilistic approaches to deterministic ones.


Uniform reliable broadcast must ensure that
\begin{inparaenum}[(i)]
\item all processes receive broadcast messages and
\item each process delivers each broadcast message once despite multiple
  receipts.
\end{inparaenum} In large and dynamic systems comprising from hundreds to
millions of processes, gossiping constitutes the \TODO{only?} scalable mean to
disseminate messages. Non-deterministic gossip-based approaches (\REF) guarantee
that each process receives each broadcast message with high probability. In this
paper, we rely on deterministic gossip-based approaches (\REF) where
peer-sampling protocols build connected graph topology. Using all provided
reliable links for broadcast ensures that each process eventually receives each
message.

Non-deterministic gossip-based approaches consist in choosing a random subset of
processes among the whole system membership to sends the broadcast message
to. Upon receipt, they forward broadcast messages similarly. However, the union
of random subsets may not include all processes of the system. Processes missing
messages must ask them to others.



This requires
\begin{itemize}
\item a structure to identify missing messages.  Interval version
  vectors~\cite{mukund2014optimized} or concise version
  vectors~\cite{malkhi2007concise} allow to identify missing messages but their
  size increases linearly and monotonically with the number of processes that
  ever broadcast a message;
\item of processes that they keep these messages as long as any process may ask
  them. In asynchronous systems, this is unbounded. Garbaging them using
  physical clocks is prone to error: 

  Processes needs to remember all messages forever. Safely garbaging them
  requires to run an overcostly consensus protocol.
\end{itemize}



In this paper, we rely on peer-sampling protocols that build connected
systems. Using all provided FIFO reliable links for broadcast guarantees that
processes eventually receive messages. However,



% Overhead: we must maintain an inview. We must send control messages. 

% Major difference, vector track the origin of the message while we track the
% direct sender. 

% We cannot antientropy with any process. (But its better that way).

%% talk about spanning tree and reliable broadcast. graph of first delivery
%% build spanning tree but we do not assume any.

Causal broadcast relies on reliable broadcast. 
% Former reliable broadcast (\REF) are probabilistic. Each process infects -- or
% gossips to -- a subset of random processes among all processes of the
% system. Processes receiving the rumour spreads it to a random
% processes. Infected process
Contrarily to former reliable broadcast protocols (\REF), \RPCBROADCAST belongs
to deterministic approaches. While non-deterministic approaches guarantee that
all processes receive each message with high probability, deterministic
approaches rely on a peer-sampling protocol that build overlay networks without
partitions to guarantee that all processes receive each message. \\
Deterministic reliable broadcast in dynamic systems either maintains
\begin{inparaenum}[(i)]
\item a set of received messages that increases linearly with the number of
  broadcast messages (\REF), or
\item a summary of received messages that increases linearly with the number of
  processes that ever broadcast a message (\REF). 
\end{inparaenum}
This restricts the usage of causal broadcast and reliable broadcast. Most
importantly, it dampens the usage of these broadcast protocols on humble devices
where local memory consumption remains an important criteria. \\
These structures allow each process to perform anti-entropy with any other
process at any time, i.e., the former identifies missing messages and already
received messages. On the opposite, processes using \RPCBROADCAST cannot perform
anti-entropy at any time. They do not maintain any structure collecting past
messages. Instead, reliable broadcast exploits the causal order property of
\RPCBROADCAST to perform anti-entropy in a window of time. As stated in
Section~\ref{subsec:complexity}, this window of time and generated traffic
remain small when the overlay network allows a form of routing. Afterwards,
broadcast messages travel normally using this link and checking if received
messages should be delivered or ignored is inexpensive.

% \TODO{Talk about early version of reliable broadcast with probabilistic guaranties?}

% It implicitly performs an anti-entropy between the two involved processes, for
% the added neighbor becomes able to identify already received messages and new
% messages among upcoming messages from the new link.

%%% Local Variables:
%%% mode: latex
%%% TeX-master: "../paper"
%%% End:
