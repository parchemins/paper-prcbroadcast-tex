
\section{Proposal}
\label{sec:proposal}

In this section, we present \RPCBROADCAST, a causal broadcast protocol removing
the last linear upper bound in terms of number of processes that ever broadcast
a message that remained on space complexity. \RPCBROADCAST exploits the
improvement brought by \PCBROADCAST on generated traffic to reduce the space
consumed by reliable broadcast, at marginal cost.

\subsection{Model}

A distributed systems comprises a set of processes that can communicate with
each other using messages.


\begin{algorithm}[h]
  \SetKwProg{Function}{function}{}{}
\SetKwProg{Upon}{upon}{}{}
\SetKwProg{Initially}{INITIALLY:}{}{}
\SetKwProg{Dissemination}{DISSEMINATION:}{}{}

\small

\DontPrintSemicolon
\LinesNumbered

%\Initially {} {
  $Q_o$ \tcp*{Out-view}
  $Q_i$ \tcp*{In-view}
  \BlankLine
  $E \leftarrow \varnothing$ \tcp*{\footnotesize Map of expected messages $Q_i : M^*$}
%}

\BlankLine

\Dissemination{}{

  \begin{multicols}{2}

  \Function{$\textup{C-broadcast}(m)$} { %\tcp*[f]{$b_p(m)$}} { 
    % $\textup{received}(m,\, \_)$ \;
    % \lForEach {$q \in Q_o$} {\textup{sendTo}($q,\, m$)}
    % \textup{R-deliver}($m$) \; % \tcp*{$d_p(m)$}
    $\textup{receive}(m,\, \_)$ \;
  }

  \BlankLine
  
  \Upon{$\textup{receive}(m,\, l)$}{
    \If {$\neg\textup{received}(m,\,l)$} {
      \lForEach {$q \in Q_o$} {\textup{sendTo}($q,\, m$)}
      % \tcp*[f]{broadcast or forward}}
      \textup{C-deliver}($m$) \; % \tcp*{$d_p(m)$}
    }
  }

  \BlankLine
  
  \Function{$\textup{received}(m,\, l)$}{
    $rcvd \leftarrow
    \exists q \in E$ \textbf{\textup{with}} $m\in E[q]$ \;
    \If {$\neg rcvd$} {
      \ForEach {$q \in Q_i$} {$E[q] \leftarrow E[q] \cup m$ \label{line:remembers}}
        % \tcp*[f]{to remember}}
    }
    $E[l] \leftarrow E[l] \setminus m$ \label{line:forgets} \;%\tcp*{to forget}
    \Return $rcvd$ \;
  }

  \end{multicols}

  \BlankLine  
}


%%% Local Variables:
%%% mode: latex
%%% TeX-master: "../paper"
%%% End:

  \caption{\label{algo:reliablebroadcast}R-broadcast at Process $p$.}
\end{algorithm}

\subsection{Operation}

Algorithm~\ref{algo:rpcbroadcast} shows the instructions of the proposed causal
broadcast. It relies on an implementation of reliable broadcast shown in
Algorithm~\ref{algo:reliablebroadcast}. When the process does not add links to
other processes, nor other processes add links to this process, causal broadcast
operation is that of reliable broadcast.

When the process wants to add a link to another process for causal broadcast, it
must make sure that this link is safe, i.e., it does not break the guaranty that
the other process delivers each message exactly once in causal order.


\begin{algorithm}[h]
  \SetKwProg{Function}{function}{}{}
\SetKwProg{Upon}{upon}{}{}
\SetKwProg{Initially}{INITIALLY:}{}{}
\SetKwProg{Safety}{SAFETY:}{}{}
\SetKwProg{Dissemination}{DISSEMINATION:}{}{}

\small

\DontPrintSemicolon
\LinesNumbered

%\begin{multicols}{2}
\Initially {} {
  $Q_o$ \tcp*{Set of processes, $p$'s outview}
  $Q_i$ \tcp*{Set of processes, $p$'s inview}
  \BlankLine  
  $B \leftarrow \varnothing$ \tcp*{link $\rightarrow$ buffered messages}
  \BlankLine  
  $S \leftarrow \varnothing$ \tcp*{Map of buffers $Q_i : M^* \times M^* \times bool$}
}

\BlankLine

\Safety{}{
  
  \Upon{$\textup{open}_o(q)$} {
    $Q_o \leftarrow Q_o \setminus q$ \;
    $\textup{sendAlpha}(p,\,q)$ \label{line:sendalpha} \tcp*{$\alpha$}
  }
  
  \Upon{$\textup{open}_i(q)$} {
    $Q_i \leftarrow Q_i \setminus q$ \;
  }

  \BlankLine
  
  \Upon{$\textup{receiveAlpha}(from,\,to)$ \tcp*[f]{$to=p$}} {
    $S[from] \leftarrow \langle \varnothing,\, \varnothing,\, false \rangle$ 
    \tcp*{$B_\alpha$}
    $\textup{sendBeta}(from,\,to)$ \label{line:sendbeta} \tcp*{$\beta$}
  }

  \BlankLine

  \Upon{$\textup{receiveBeta}(from,\,to)$ \tcp*[f]{$from=p$}} {
    $B[to] \leftarrow \varnothing$ \tcp*{$B_\beta$}
    $\textup{sendPing}(from,\, to)$ \label{line:sendpi} \tcp*{$\pi$}
  }
  
  \BlankLine

  \Upon{$\textup{receivePing}(from,\,to)$ \tcp*[f]{$to=p$}} {
    \textbf{let} $\langle B_\alpha ,\, B_\pi ,\, \_ \rangle \leftarrow S[from]$ \;
    $S[from] \leftarrow \langle B_\alpha,\, B_\pi,\, true \rangle$ \tcp*{$B_\pi$}
    $\textup{sendPong}(from,\, to)$ \label{line:sendrho} \tcp*{$\rho$}
  }

  \BlankLine

  \Upon{$\textup{receivePong}(from,\,to)$ \tcp*[f]{$from=p$}} {
    $\textup{sendBuffer}(from,\,to,\, B[to])$ \;
    $B \leftarrow B \setminus to$ \;
    $Q_o \leftarrow Q_o \cup to$
  }

  \BlankLine

  \Upon{$\textup{receiveBuffer}(from,\, to,\, B_\beta)$} {
    \textbf{let} $\langle B_\alpha,\, B_\pi,\, \_ \rangle \leftarrow S[from]$ \;
%    \textbf{let} $potential \leftarrow buf \setminus B_\alpha$ \;
    \ForEach {$m \in B_\beta\setminus B_\alpha \setminus B_\pi$ }
    {$\textup{receive}(m,\,from)$ \tcp*[f]{to deliver}}
    $D[from] \leftarrow B_\pi \setminus (B_\beta\setminus B_\alpha)$ \tcp*{to expect}
    $Q_i \leftarrow Q_i \cup from$ 
  }
  
  \BlankLine

  \Upon{$\textup{close}_o(q)$} {
    $B \leftarrow B \setminus q$
  }
  \Upon{$\textup{close}_i(q)$} {
    $S \leftarrow S \setminus q$
  }

  
}

\BlankLine

\Dissemination{}{
  
  \Function{$\RPCBROADCAST(m)$} { %\tcp*[f]{$b_p(m)$}} {
    $\textup{buffering}(m)$ \;
    $\textup{R-broadcast}(m)$
  }

  \BlankLine
  
  \Upon{$\textup{R-deliver}(m)$} {
    $\textup{buffering}(m)$ \;
    $\textup{RPC-deliver}(m)$
  }

  \BlankLine

  \Function{$\textup{buffering}(m)$}{ 
    \lForEach {$q \in B$} {$B[q] \leftarrow B[q] \cup m$}
    \ForEach {$q \in S$} {
      \textbf{let} $\langle B_\alpha,\, B_\pi,\, received_\pi \rangle \leftarrow S[q]$ \;
      \lIf {$received_\pi$} {$B_\pi \leftarrow B_\pi \cup m$}
      \lElse {$B_\alpha \leftarrow B_\alpha \cup m$} 
    }
  }
}
%\end{multicols}

%%% Local Variables:
%%% mode: latex
%%% TeX-master: "../paper"
%%% End:

  \caption{\label{algo:rpcbroadcast}RPC-broadcast at Process $p$.}
\end{algorithm}


\begin{figure*}
  \begin{center}
    
\begin{tikzpicture}[scale=1]

  \small
  
  \newcommand\X{1.7*\columnwidth/8pt};
  \newcommand\YA{0pt};
  \newcommand\YB{-60pt};


  \draw[->, thick](0*\X, \YA) -- (8*\X, \YA);
  \draw[->, thick](0*\X, \YB) -- (8*\X, \YB);
  
  \draw[fill=white] (0*\X, \YA) node{\textbf{\textup{A}}} +(-5pt, -5pt) rectangle +(5pt, 5pt);
  \draw[fill=white] (0*\X, \YB) node{\textbf{\textup{B}}} +(-5pt, -5pt) rectangle +(5pt, 5pt);

  \draw ( 1*\X, \YA ) node[above]{$\mathcal{A}$};
  \draw ( 1*\X, \YB ) node[below]{$\mathcal{B}$};

  \draw[densely dashed,->] ( 2*\X, \YA ) -- node[sloped, above]{$\alpha$} (3*\X, \YB)
  node[below left]{$\mathcal{A}$};

  \draw[decorate,decoration={brace,amplitude=6pt,mirror,raise=4pt}] (3*\X,
  -5+\YB) -- node[anchor=north, yshift=-10pt]{$B_\alpha$}
  (5*\X, -5+\YB);

  \draw ( 3*\X, \YA ) node[above]{$\mathcal{C}$};

  \draw[densely dashed,->] ( 3*\X, \YB ) -- node[sloped, above]{$\beta$} (4*\X, \YA)
  node[above left]{$\mathcal{B}$};

  \draw[decorate,decoration={brace,amplitude=6pt,raise=4pt}] (4*\X,
  5+\YA) -- node[anchor=south, yshift=10pt]{$B_\beta$}
  (6*\X, 5+\YA);
  
  
  \draw ( 4*\X, \YB ) node[below]{$\mathcal{D}$};

  \draw[densely dashed,->] ( 4*\X, \YA ) -- node[sloped, above]{$\pi$} (5*\X, \YB)
  node[below left]{$\mathcal{C}$};

  \draw[decorate,decoration={brace,amplitude=6pt,mirror,raise=4pt}] (5*\X,
  -5+\YB) -- node[anchor=north, yshift=-10pt]{$B_\pi$}
  (7*\X, -5+\YB);


  \draw ( 5*\X, \YA ) node[above]{$\mathcal{E}$};

  \draw[densely dashed,->] ( 5*\X, \YB ) -- node[sloped, above]{$\rho$} (6*\X, \YA)
  node[above left]{$\mathcal{D}$};
  
  \draw (6*\X, \YB) node[below]{$\mathcal{F}$};

  \draw[->, very thick] ( 6*\X, \YA ) -- node[sloped, above]{$\bm{B_\beta}$} (7*\X, \YB);
%  node[below left]{$\mathcal{E}$};


\end{tikzpicture}

%%% Local Variables:
%%% mode: latex
%%% TeX-master: "../paper"
%%% End:

    \caption{Meow.}
  \end{center}
\end{figure*}

\subsection{Complexity}


%%% Local Variables:
%%% mode: latex
%%% TeX-master: "../paper"
%%% End:
