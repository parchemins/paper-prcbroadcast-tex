
\section{Proposal}
\label{sec:proposal}

In this section, we present \RPCBROADCAST, a causal broadcast protocol removing
the last linear upper bound in terms of number of processes that ever broadcast
a message that remained on space complexity. \RPCBROADCAST exploits the
improvement brought by \PCBROADCAST on generated traffic to reduce the space
consumed by reliable broadcast, at marginal cost.

\subsection{Model}

A distributed system comprises a set of processes that can communicate with each
other using messages. Processes may not have full knowledge of the membership of
the system. Instead, processes build and maintain overlay networks: each process
updates a local partial view of logical communication channels, i.e., a set of
processes to communicate with. The partial view is usually much smaller than the
actual system size. We speak of overlay networks, networks, or distributed
systems indifferently.

\begin{definition}[Overlay network]
  An overlay network comprises a set of processes that run a set of instructions
  sequentially.  An overlay network also comprises a set of directed
  links. \\
  An overlay network is dynamic if the set of processes and the set of links is
  mutable. Otherwise, the overlay network is static. \\
  The set of links departing from a process is its neighborhood, or partial
  view, or out-view. The set of links arriving to a process is its in-view. \\
  A link from a process to another process allows the former to transmit
  information to the latter. Processes transmit information using asynchronous
  message passing.\\
  Processes are faulty if they crash, otherwise they are correct. There are no
  byzantine processes.
\end{definition}

Processes can communicate with each other using message passing. Process~A can
send a message $m$ to Process~B $s_{AB}(m)$. Process~A can receive a message $m$
from another process B $r_{AB}(m)$, or from any other process
$r_A(m)$. Process~A can send a message $m$ to all other processes of the system,
i.e., it broadcasts a message $b_A(m)$. Process~A can deliver a message
$d_A(m)$.


\begin{definition}[Uniform reliable broadcast] 
  When a process broadcasts a message to all processes of the network, correct
  processes eventually receive it. Uniform reliable broadcast guarantees 3
  properties:
  \begin{itemize}
  \item Validity: If a correct process broadcasts a message, then it
    eventually delivers it.
  \item Uniform Agreement: If a process -- correct or not -- delivers a message,
    then all correct processes eventually deliver it.
  \item Uniform Integrity: A process delivers a message at most once, and only if
    it was previously broadcast.
  \end{itemize}
\end{definition}

\begin{algorithm}[h]
  \SetKwProg{Function}{function}{}{}
\SetKwProg{Upon}{upon}{}{}
\SetKwProg{Initially}{INITIALLY:}{}{}
\SetKwProg{Dissemination}{DISSEMINATION:}{}{}

\small

\DontPrintSemicolon
\LinesNumbered

\Initially {} {
  $Q_o$ \tcp*{Set of processes, $p$'s out-view}
  $Q_i$ \tcp*{Set of processes, $p$'s in-view}
  \BlankLine
  $E \leftarrow \varnothing$ \tcp*{Map of expected messages $Q_i : M^*$}
}

\BlankLine

\Dissemination{}{

%  \begin{multicols}{2}

  \Function{$\textup{R-broadcast}(m)$} { %\tcp*[f]{$b_p(m)$}} { 
    % $\textup{received}(m,\, \_)$ \;
    % \lForEach {$q \in Q_o$} {\textup{sendTo}($q,\, m$)}
    % \textup{R-deliver}($m$) \; % \tcp*{$d_p(m)$}
    $\textup{receive}(m,\, \_)$ \;
  }

  \BlankLine
  
  \Upon{$\textup{receive}(m,\, l)$}{
    \If {$\neg\textup{received}(m,\,l)$} {
      \lForEach {$q \in Q_o$} {\textup{sendTo}($q,\, m$)}
      % \tcp*[f]{broadcast or forward}}
      \textup{R-deliver}($m$) \; % \tcp*{$d_p(m)$}
    }
  }

  \BlankLine
  
  \Function{$\textup{received}(m,\, l)$}{
    \textbf{let} $rcvd \leftarrow
    \exists q \in E$ \textbf{\textup{with}} $m\in E[q]$ \;
    \If {$\neg rcvd$} {
      \ForEach {$q \in Q_i$} {$E[q] \leftarrow E[q] \cup m$ \label{line:remembers}}
        % \tcp*[f]{to remember}}
    }
    $E[l] \leftarrow E[l] \setminus m$ \label{line:forgets} \;%\tcp*{to forget}
    \Return $rcvd$ \;
  }

%  \end{multicols}

%  \BlankLine  
}


%%% Local Variables:
%%% mode: latex
%%% TeX-master: "../paper"
%%% End:

  \caption{\label{algo:reliablebroadcast}R-broadcast at Process $p$.}
\end{algorithm}

Algorithm~\ref{algo:reliablebroadcast} shows the instructions of a uniform
reliable broadcast that purges its local structure over time but only in static
networks. The implementation uses the in-view and exploits the property that
each process sends each message exactly once to each neighbor in their
out-view. It associates with each link from the in-view a set of awaited
messages. When Process~A receives a message for the first time from Process~B,
it awaits a copy of this message from all other processes in its in-view. Once
it receives this copy, it removes the message from the set of awaited
messages. Process~A will never receive a copy of this message from this
link. Once it received all awaited copies, it will never receive a copy of this
message. The principle of this implementation is similar to that of
Figure~\ref{fig:generalpurge} to which it adds an awareness of links with
awaited messages. Using this reliable broadcast implementation, each process
delivers each message exactly once.

In addition to reliable delivery, broadcast protocols can guarantee a delivery
order of messages. To define a delivery order among messages, we define time in
a logical sense using Lamport’s definition. 

\begin{definition}[Happen before~\cite{lamport1978time}]
  Happen before is a transitive, irreflexive, and antisymmetric relation that
  defines a strict partial orders of events. The sending of a message always
  precedes its receipt.
\end{definition}

\begin{definition}[Causal order]
  The delivery order of messages follows the happen before relationships of the
  corresponding broadcasts.
\end{definition}


\subsection{Operation}

Algorithm~\ref{algo:rpcbroadcast} shows the instructions of the proposed causal
broadcast. It relies on an implementation of reliable broadcast shown in
Algorithm~\ref{algo:reliablebroadcast}. When the process does not add links to
other processes, nor other processes add links to this process, causal broadcast
operation is that of reliable broadcast.

When the process wants to add a link to another process for causal broadcast, it
makes sure that this link cannot break the guaranty that the other process
delivers each message exactly once in causal order.


\begin{algorithm}[h]
  \SetKwProg{Function}{function}{}{}
\SetKwProg{Upon}{upon}{}{}
\SetKwProg{Initially}{INITIALLY:}{}{}
\SetKwProg{Safety}{SAFETY:}{}{}
\SetKwProg{Dissemination}{DISSEMINATION:}{}{}

\SetKwComment{EmptyComment}{}{}

\small

\DontPrintSemicolon
\LinesNumbered

%\begin{multicols}{2}
\Initially {} {
%  $Q_o$ \tcp*{Set of processes, $p$'s outview}
%  $Q_i$ \tcp*{Set of processes, $p$'s inview}
%  \BlankLine  
  $B \leftarrow \varnothing$ \tcp*{$@$Sender Map of buffers $Q_o : M^*$}
%  \BlankLine  
  $S \leftarrow \varnothing$ \tcp*{$@$Receiver Map of buffers $Q_i : M^* \times M^* \times bool$}
}

\BlankLine

\Dissemination{}{

  \begin{multicols}{2}
  \Function{$\RPCBROADCAST(m)$} { %\tcp*[f]{$b_p(m)$}} {
%    $\textup{buffering}(m)$ \;
    $\textup{R-broadcast}(m)$
  }

%  \BlankLine
  
  \Upon{$\textup{R-deliver}(m)$} {
    $\textup{buffering}(m)$ \;
    $\textup{PRC-deliver}(m)$
  }

%  \BlankLine

  \Function{$\textup{buffering}(m)$}{ 
    \lForEach {$q \in B$} {$B[q] \leftarrow B[q] \cup m$}

    \ForEach{$\langle B_\alpha,\, B_\pi,\, received_\pi\rangle \in S$}{
      \lIf{$received_\pi$}{$B_\pi \leftarrow B_\pi \cup m$}
      \lElse{$B_\alpha \leftarrow B_\alpha \cup m$}
    }
    
    %   $S[q] = \langle B_\alpha,\, \_,\, false \rangle$}
    % {$B_\alpha \leftarrow B_\alpha \cup m$}

    % \lForEach{$q \in S$ \textup{\textbf{such that}}
    %   $S[q] = \langle\, \_,\,B_\pi,\, true \rangle$}
    % {$B_\pi \leftarrow B_\pi \cup m$}

  }
  \end{multicols}
}

\BlankLine

\Safety{}{
  \begin{multicols}{2}
    \EmptyComment*[l]{\uline{$@$Sender}}
  \Upon{$\textup{open}_o(to)$} {
    $Q_o \leftarrow Q_o \setminus to$  \; % \tcp*{unsafe to send to $to$}
    $\textup{send-}\alpha(p,\,q)$ \label{line:sendalpha} \; %  \tcp*{$\alpha$}
  }

  \EmptyComment*[l]{\uline{$@$Receiver}}
  \Upon{$\textup{open}_i(from)$} {
    $Q_i \leftarrow Q_i \setminus from$ \; % \tcp*{unsafe to receive from $from$}
  }
  \end{multicols}

  \BlankLine
  
  \begin{multicols}{2}
  \Upon{$\textup{receive-}\beta(from,\,to)$}{% \tcp*[f]{$from=p$}} {
    $B[to] \leftarrow \varnothing$ \; %\tcp*{initialize $B_\beta$}
    $\textup{send-}\pi(from,\, to)$ \label{line:sendpi} \; % \tcp*{$\pi$}
  }

  \Upon{$\textup{receive-}\alpha(from,\,to)$}{ % \tcp*[f]{$to=p$}} {
    $S[from] \leftarrow \langle \varnothing,\, \varnothing,\, false \rangle$ \;
    %% \tcp*{initialize $B_\alpha$}
    $\textup{send-}\beta(from,\,to)$ \label{line:sendbeta} \; % \tcp*{$\beta$}
  }

  \end{multicols}
  
  \BlankLine
  
  \begin{multicols}{2}
  \Upon{$\textup{receive-}\rho(from,\,to)$}{% \tcp*[f]{$from=p$}} {
    $\textup{send-}B_\beta(from,\,to,\, B[to])$ \label{line:sendbuffer}\;
    $B \leftarrow B \setminus to$ \;
    $Q_o \leftarrow Q_o \cup to$ \; %\tcp*{safe to send to $to$}
  }

  \Upon{$\textup{receive-}\pi(from,\,to)$}{ % \tcp*[f]{$to=p$}} {
    \textbf{let} $\langle B_\alpha ,\, B_\pi ,\, \_ \rangle \leftarrow S[from]$ \;
    $S[from] \leftarrow \langle B_\alpha,\, B_\pi,\, true \rangle$ \; % \tcp*{initialize $B_\pi$}
    $\textup{send-}\rho(from,\, to)$ \label{line:sendrho} \; % \tcp*{$\rho$}
  }
  \end{multicols}

  \BlankLine
  
  \begin{multicols}{2}
    \EmptyComment*{}
    \EmptyComment*{}
    \EmptyComment*[r]{filter messages to ignore $\rightarrow$\hspace{2.2em}}
    \EmptyComment*[r]{to deliver $\rightarrow$\hspace{2.2em}}
    \EmptyComment*[r]{to expect $\rightarrow$\hspace{2.2em}}
    \columnbreak
  \Upon{$\textup{receive-}B_\beta(from,\, to,\, B_\beta)$} {
    \textbf{let} $\langle B_\alpha,\, B_\pi,\, \_ \rangle \leftarrow S[from]$ \;
%    \textbf{let} $potential \leftarrow buf \setminus B_\alpha$ \;
    \ForEach {$m \in B_\beta\setminus B_\alpha \setminus B_\pi$ }
    {$\textup{receive}(m,\,from)$ \label{line:todeliver}}  %$ \tcp*[f]{to deliver}}
    $E[from] \leftarrow B_\pi \setminus (B_\beta\setminus B_\alpha)$ \label{line:toexpect} \;% \tcp*{to expect}
    $S \leftarrow S \setminus from$ \;
    $Q_i \leftarrow Q_i \cup from$ \; % \tcp*{safe to receive from $from$}
  }
  \end{multicols}
  \BlankLine
  
  \begin{multicols}{2}
  \Upon{$\textup{close}_o(to)$} {
    $B \leftarrow B \setminus to$
  }
  \Upon{$\textup{close}_i(from)$} {
    $S \leftarrow S \setminus from$ \;
    $E \leftarrow E \setminus from$
  }
  \end{multicols}
  \BlankLine
}


%%% Local Variables:
%%% mode: latex
%%% TeX-master: "../paper"
%%% End:

  \caption{\label{algo:rpcbroadcast}RPC-broadcast at Process $p$.}
\end{algorithm}


\begin{figure*}
  \begin{center}
    \input{input/figtimelinerpcbroadcast.tex}
    \caption{Meow.}
  \end{center}
\end{figure*}

\subsection{Complexity}


%%% Local Variables:
%%% mode: latex
%%% TeX-master: "../paper"
%%% End:
