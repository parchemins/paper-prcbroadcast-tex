
\section{Related work}
\label{sec:relatedwork}

\begin{table*}
  \begin{center}
    \caption{\label{table:complexity} Complexity of broadcast algorithms at each
      process. $N$ the number of processes that ever broadcast a message. $P$
      the number of processes in the system. $W$ the number of messages received
      but not delivered yet. $Q_i$ is the number of incoming links. $M$ is the
      number of messages already delivered that will be received again from at
      least one link in $Q_i$.}
  \newcommand{\cmark}{\ding{51}}%
\newcommand{\xmark}{\ding{55}}%

\setlength{\tabcolsep}{4pt} % General space between cols (6pt standard)

\small

\begin{tabularx}{0.98\columnwidth}{@{}Xcccc@{}}
  & \makecell{message\\overhead} &  \makecell{delivery\\execution time} & \makecell{local space\\consumption} & \makecell{\# control messages\\per added link} \\ \cmidrule{2-5}
  reliable broadcast~\cite{hadzilacos1994modular} & $O(1)$ & $O(1)$ & $O(N)$ & $0$ \\
  vector-based causal broadcast~\cite{schwarz1994detecting} & $O(N)$ & $O(W.N)$ & $O(N+W.N)$ & $0$ \\ 
  \PCBROADCAST (\REF) & $O(1)$ & $O(1)$ & $O(N)$ & $3$ to $2P^2$ \\ \hline\hline
  \textbf{this paper (\RPCBROADCAST)} & $O(1)$ & $O(Q_i)$ & $\mathbf{O(Q_i.M)}$ & $\mathbf{6}$ to $\mathbf{4P^2}$ \\
%  \bottomrule
\end{tabularx}

%%% Local Variables:
%%% mode: latex
%%% TeX-master: "../paper"
%%% End:

  \end{center}
\end{table*}

Causal broadcast ensures causal order and forbids multiple
delivery. \RPCBROADCAST uses the former to improve on the complexity of the
latter.  This section reviews state-of-the-art implementations of broadcast
protocols that forbid multiple delivery in asynchronous and dynamic systems.

% The latter property similar to those of uniform reliable
% broadcast~\cite{hadzilacos1994modular}.

% Causal broadcast protocols rely on reliable
% broadcast~. Uniform reliable broadcast must ensure
% that
% \begin{inparaenum}[(i)]
% \item all processes receive broadcast messages and
% \item each process delivers each broadcast message once despite multiple
%   receipts.
% \end{inparaenum} 
% This extends from probabilistic
% dissemination approaches to deterministic ones; from physical clock techniques
% to vector-based approaches.

% Gossiping constitutes the most efficient mean to disseminate messages to all
% processes in dynamic systems comprising from hundreds to millions of processes.
% Gossiping distributes the load of dissemination among processes. Every process
% maintains a partial view of the system and broadcasts or forwards messages using
% these links. Assuming a strongly connected overlay network, all messages reach
% all processes of the system either directly or transitively. While it ensures
% the receipt of messages, it does not forbid multiple delivery.

Building \textbf{specific dissemination topologies} such as tree or ring
guarantees that every process receives each message
once~\cite{bravo2017saturn,raynal2013distributed}. Processes deliver messages as
soon as they arrive. They do not need to save any control information about
messages, for they will never receive a copy of this message again. While these
approaches are lightweight, they stay confined to systems where failures are
uncommon, and where churn rate remains
low~\cite{krasikova2016hashtable}. \RPCBROADCAST generalizes on these specific
topologies. It follows the same principle where the topology impacts on the
number of receipts. Its space complexity scales linearly with this number of
receipts. In turns, \RPCBROADCAST inherits from the resilience of the underlying
topology maintained by processes. \RPCBROADCAST supports dynamic systems without
assuming any specific topology.


% In this paper, we focus on deterministic overlay
% We focus on deterministic gossip-based
% broadcast~\cite{friedman2004causal,nedelec2017adaptive} where peer-sampling
% protocols build connected graph topology. Using all provided reliable links for
% broadcast ensures that each process eventually receives each message. A process
% keeps a message as long as one of its neighbor did not receive it. Afterwards,
% it safely removes it.


% \textbf{Non-deterministic} gossip-based
% approaches~\cite{birman1999bimodal,demers1987epidemic} guarantee that each
% process receives each broadcast message with high probability. Processes must be
% able to identify and retrieve missing messages by themselves. 


Without any specific dissemination topology, each process may receive each
broadcast message multiple times. Despite multiple receipts, a process must
deliver a message once.
% When multiple receipts can occur, each process maintains a local structure to
% detect and discard copies of broadcast messages. Using \textbf{physical
%   clocks~\cite{cachin2011introduction,demers1987epidemic}} allows removing
% control information over time. Control information is considered obsolete above
% an age threshold. The structure is non-monotonic. However, defining the age
% threshold is error prone in asynchronous systems. Set too small and processes
% may suffer multiple delivery with cascading effects as in
% Figure~\ref{fig:memorylinkfailsD}.  \RPCBROADCAST maintains a non-monotonic
% structure but makes sure that it removes all and only obsolete control
% information.
Using local structures based on \textbf{logical clocks~\cite{lamport1978time}},
every process differentiates between the first receipt of a broadcast message
and the additionnal receipts of this message. It allows to deliver the former
while ignoring the latter. Unfortunately, the size of these structures increases
monotonically and linearly with the number of processes that ever broadcast a
message~\cite{malkhi2007concise,mukund2014optimized}.  Processes cannot reclaim
the space consumed, for it would require running an overcostly distributed
garbage collection that is equivalent to a distributed
consensus~\cite{abdullahi1998garbage}.  This limits their use to context where
the number of broadcasters is known to be small.  \RPCBROADCAST uses local
structures based on logical clocks too. However, instead of saving the past
deliveries of broadcasters, its saves the messages expected from direct
neighbors. The set of expected messages varies over receipts, and the number of
neighbors can be far smaller than the set of broadcasters. \RPCBROADCAST scales
in large and dynamic systems. Among others, \RPCBROADCAST fits contexts where
the number of participants is unknown, such as distributed collaborative
editing~\cite{nedelec2016crate}.

% It is worth noting that  logical clocks allow each process to perform
% anti-entropy with any other process at any time. The former identifies its
% missing messages compared to the latter.  The latter sends them to keep
% consistent states. This is useful when applications can go
% offline~\cite{demers1987epidemic}. \RPCBROADCAST implicitly performs
% anti-entropy on-demand to initialize memory link but cannot provide it as a
% feature for applications. If required, this is the responsibility of
% applications to allow anti-entropy (e.g. using Merkle
% tree~\cite{merkle1988digital}).

%%% Local Variables:
%%% mode: latex
%%% TeX-master: "../paper"
%%% End:
