
\section{Related Work}
\label{sec:relatedwork}

% Overhead: we must maintain an inview. We must send control messages. 

% Major difference, vector track the origin of the message while we track the
% direct sender. 

% We cannot antientropy with any process. (But its better that way).

Causal broadcast relies on reliable broadcast. Reliable broadcast in dynamic
systems either maintains
\begin{inparaenum}[(i)]
\item a set of received messages that increases linearly with the number of
  broadcast messages (\REF), or
\item a summary of received messages that increases linearly with the number of
  processes that ever broadcast a message (\REF). 
\end{inparaenum}
This restricts the usage of causal broadcast and reliable broadcast. Most
importantly, it dampens the usage of these broadcast protocols on humble devices
where local memory consumption remains an important criteria. \\
These structures allow each process to perform anti-entropy with any other
process at any time, i.e., the former identifies missing messages and already
received messages. On the opposite, processes using \RPCBROADCAST cannot perform
anti-entropy at any time. They do not maintain any structure collecting past
messages. Instead, reliable broadcast exploits the causal order property of
\RPCBROADCAST to perform anti-entropy in a window of time. As stated in
Section~\ref{subsec:complexity}, this window of time and generated traffic
remain small when the overlay network allows a form of routing. Afterwards,
broadcast messages travel normally using this link and checking if received
messages should be delivered or ignored is inexpensive.

\TODO{Talk about early version of reliable broadcast with probabilistic guaranties?}

% It implicitly performs an anti-entropy between the two involved processes, for
% the added neighbor becomes able to identify already received messages and new
% messages among upcoming messages from the new link.

%%% Local Variables:
%%% mode: latex
%%% TeX-master: "../paper"
%%% End:
