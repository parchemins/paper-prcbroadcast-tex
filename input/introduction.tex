 
\section{Introduction}

Causal broadcast constitutes the core communication primitive of many
distributed systems~\cite{hadzilacos1994modular}. Applications such as
distributed social networks~\cite{borthakur2013petabyte}, distributed
collaborative software~\cite{heinrich2012exploiting,nedelec2016crate}, or
distributed data
stores~\cite{bailis2013bolton,bravo2017saturn,demers1987epidemic,lloyd2011cops,shapiro2011comprehensive}
use causal broadcast to ensure consistency criteria.  Causal broadcast ensures
reliable receipt of broadcast messages, exactly-once delivery, and causal
delivery following Lamport's happen before
relationship~\cite{lamport1978time}. When Alice comments on Bob's picture,
nobody sees Alice's comment without Bob's picture, and nobody sees multiple
occurrences of Alice's comment or Bob's picture.

Vector clock-based
approaches~\cite{malkhi2007concise,mukund2014optimized,nedelec2018pcbroadcast}
need to keep all their control information forever. They cannot forget any
control information.  The consumed memory monotonically increases with the
number of processes that ever broadcast a message $O(N)$.  They become
unpractical in large and dynamic system comprising from hundreds to millions of
processes joining, leaving, self-reconfiguring, or crashing at any time.

In this paper, we propose a novel implementation of causal broadcast that
forgets all and only obsolete control information. A process $p$ that has $i$
incoming links receives each message $i$ times. A message $m$ is active for $p$
between its first and last reception by $p$. Process $p$ keeps control
information about all its active messages. As soon as a message becomes
inactive, the process can forget all control information related to it.
Consequently, processes do no store any permanent control information about
messages. When no message is active, no control information is stored in the
system. 

\noindent Our contribution is threefold:
\begin{itemize}[leftmargin=*]
\item We define the notion of \emph{link memory} as a mean for each process to
  forbid multiple delivery.  Link memory allows each process to identify
  processes from which it will receive a copy of an already delivered
  message. This allows each process to safely remove obsolete control
  information about broadcast messages that will never be received again. We
  prove that using a causal delivery, each process can build such knowledge even
  in dynamic systems where processes may join, leave, self-reconfigure, or crash
  at any time.
\item We propose an implementation of causal broadcast that uses the notion of
  link memory, where each process manages a local data the size of which is
  $O(i \cdot A)$ where $i$ is the number of incoming links and $A$ is the number
  of active messages. Moreover, the only control information piggybacked on
  messages is a scalar Lamport clock.
\item We evaluate our implementation using large scale simulations. The
  experiments highlight the space consumed and the traffic generated by our
  protocol in dynamic systems with varying latency. The results confirm that the
  proposed approach scales with system settings and use.
\end{itemize}

The rest of this paper is organized as follows.  Section~\ref{sec:proposal}
describes the model, highlights the issue, introduces the principle solving the
issue, provides an implementation solving the issue along with its complexity
analysis. Section~\ref{sec:experimentation} shows the
experiments. Section~\ref{sec:relatedwork} reviews related work. We conclude and
discuss about perspectives in Section~\ref{sec:conclusion}.


%%% Local Variables:
%%% mode: latex
%%% TeX-master: "../paper"
%%% End:
