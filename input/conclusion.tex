
\section{Conclusion}
\label{sec:conclusion}

In this paper, we proposed a causal broadcast implementation that provides a
novel trade-off between speed, memory, and traffic.  Our approach exploits
causal order to improve on the space complexity of the implementation that
forbids multiple delivery.  The local space complexity of this protocol does not
monotonically increase and depends at each moment on the number of messages
still in transit and the degree of the communication graph.
The overhead in terms of number of control messages depends on the dynamicity of
the system and remains low upon the assumption that the overlay network allows a
form of routing.
This advantageous trade-off makes causal broadcast a lightweight and efficient
middleware for group communication in distributed systems. 

As future work, we plan to investigate on ways to retrieve the partial order of
messages out of \RPCBROADCAST.
Applications may require more than causal order, they also may need to identify
concurrent messages~\cite{sun2009contextbased}. \RPCBROADCAST discards a lot of
information by ignoring multiple receipts altogether. Analyzing the receipt
order could provide insight on the partial order. The cost could depend on the
actual concurrency of the system.

%%% Local Variables:
%%% mode: latex
%%% TeX-master: "../paper"
%%% End:
