
\section{Issues and motivations}
\label{sec:motivations}

A distributed system is set of processes that can communicate with each other
using messages (\REF). Processes may only have partial knowledge of the full
system membership, for maintaining full knowledge is overcostly in large and
dynamic systems where processes can join, or leave at any time (\REF). A process
sends a message to all processes that belong to the system by
broadcasting. Reliable broadcast ensures that all messages reach all processes
and processes deliver each message exactly once; and causal broadcast is a
reliable broadcast that ensures causal order on message delivery (\REF).

% Causal broadcast is a communication primitive that allows a process to send
% messages to all processes of its distributed system (\REF). Message deliveries
% follow the happen before relationship (\REF). If the sending of a message $m$
% precedes the sending of a message $m'$ then all processes that deliver these two
% messages need to deliver $m$ before $m'$. Otherwise they deliver them in any
% order. Each process may receive a message multiple times but it delivers it
% exactly once.

Implementing causal broadcast notoriously required vector clocks to ensure
causal order and reliable delivery (\REF). The size of vector clocks increases
linearly with the number of processes that ever broadcast a message.  This
negatively impacts generated traffic and local memory consumption as well.  In
the worst case, a system comprising 1 million processes requires that each
process maintains a local vector of 1 million entries and sends it within each
broadcast message. This forbids the usage of causal broadcast in large and
dynamic systems where any process could participate, such as in distributed
collaborative editing (\REF). Fortunately, \PCBROADCAST (\REF) partially removed
the need for such vectors. Instead of piggybacking control information in
messages to ensure causal order, \PCBROADCAST uses FIFO communication links,
gossiping, and few control messages (see Table~\ref{table:complexity}). Yet,
vectors remain locally to ensure that despite multiple receipts, each process
delivers each message exactly once. This constitutes the last monotonic and
linear upper bound remaining for causal broadcast.


% \begin{figure*}
%   \begin{center}
%     \subfloat[Part A][\label{fig:spaceproblemA}Process~B broadcasts $b$ along
%     with control information $\langle B,\,1 \rangle$.]
%     {\input{input/figspaceproblemA.tex}}
%     \hspace{10pt}
%     \subfloat[Part B][\label{fig:spaceproblemB}Process~A receives, saves in its
%     local vector, delivers and forwards
%     $b$. Process~B wants to add Process~C in its direct neighbors for
%     causal broadcast. It sends a control message $\pi$ to Process~C using 
%     Process~A as mediator.]
%     {\input{input/figspaceproblemB.tex}}    
%     \hspace{10pt}
%     \subfloat[Part C][\label{fig:spaceproblemC}Process~A broadcasts $a$ along with
%     control information $\langle A,\, 1 \rangle$. Then, it routes $\pi$ towards
%     Process~C. Process~B receives $b$ but discards it, for it is already
%     registered in Process~B's local vector.]
%     {\input{input/figspaceproblemC.tex}}
%     \hspace{10pt}
%     \subfloat[Part D][\label{fig:spaceproblemD}Process~B and Process~C receive,
%     save in their local vector, deliver and forward $a$. In addition, Process~B
%     buffers $a$ to send it later to Process~C.]
%     {\input{input/figspaceproblemD.tex}}
%     \hspace{10pt}
%     \subfloat[Part E][\label{fig:spaceproblemE}Process~C receives $\pi$ and 
%     replies $\rho$ to Process~B.]
%     {
\begin{tikzpicture}[scale=1]
  
  \small
  
  \newcommand\X{210/5pt};
  \newcommand\Y{30pt};

  
  \draw[fill=white] (0*\X, 0*\Y) node{\textbf{A}} +(-5pt, -5pt) rectangle +(5pt, 5pt);
  \draw (-5+0*\X, 5+0*\Y) node[above]{$[A:1,\,B:1]$};
  \draw[fill=white] (1*\X, -1*\Y) node{\textbf{B}} +(-5pt, -5pt) rectangle +(5pt, 5pt);
  \draw (1*\X, -5-1*\Y) node[below]{$[A:1,\,B:1]$};
  \draw[fill=white] (2*\X,  0*\Y) node{\textbf{C}} +(-5pt, -5pt) rectangle +(5pt, 5pt);
  \draw (5+2*\X, 5+0*\Y) node[above]{$[A:1,\,B:1]$};
  
  \draw[->](5+0*\X, 0*\Y) -- 
  (-5+1*\X, -1*\Y); %% A->B

  \draw[<-](5+0*\X, -5+0*\Y) --
  (-5+1*\X, -5-1*\Y); %% A<-B
  
  \draw[->](5+0*\X, 5+0*\Y) --
  (-5+2*\X, 5+0*\Y); % A->C
  
  \draw[<-](5+0*\X,  1.25+ 0*\Y) --
  (-5+2*\X,  1.25+ 0*\Y); % A<-C
  
  % \draw[->,dashed](5+1*\X, -1*\Y) -- (-5+2*\X, 0*\Y); %% B<-C
 \draw[<-,densely dotted,decorate, decoration={snake, amplitude=0.3mm}](5+5+1*\X, 5+-1*\Y) -- 
 node[above, sloped]{$\bm{\rho}$}
 (-5+2*\X, 0*\Y);

  \draw[->, dashed](5+1*\X, -5-1*\Y) --
  node[sloped, below]{$[a_{A,\,1}]$} (-5+2*\X, -5+0*\Y); %% B->C



\end{tikzpicture}}
%     \hspace{10pt}
%     \subfloat[Part F][\label{fig:spaceproblemF}Process~B empties its buffer to
%     Process~C. The latter will discard the message $a$, for it is already registered
%     in Process~C's local vector]
%     {\input{input/figspaceproblemF.tex}}
%     \caption{\label{fig:spaceproblem}\PCBROADCAST relies on vectors to deliver
%       once and discard multiple receipts. The size of vectors increases linearly
%       with the number of processes that ever broadcast a message.}
%   \end{center}
% \end{figure*}

% Figure~\ref{fig:spaceproblem} depicts the functioning of \PCBROADCAST while
% highlighting the issue. In this scenario, the system comprises 3
% processes. Processes maintain a vector of pairs
% $\langle site,\, counter \rangle$ summarizing the received messages. At first,
% vectors start empty and Process~B cannot communicate directly with Process~C.
% In Figure~\ref{fig:spaceproblemA}, Process~B broadcasts $b$. It assigns it a
% unique identifier $\langle B,\, 1 \rangle$ and saves this element in its local
% vector. In Figure~\ref{fig:spaceproblemB}, Process~A receives $b$. Its vector
% does not contain this element. Hence, Process~A saves $\langle B,\,1\rangle$ in
% its vector, delivers and forwards $b$. In the meantime, Process~B wants to add a
% direct communication link to Process~C for causal broadcast. For the sake of
% safety, i.e., sending messages using this new link cannot violate causal order,
% Process~B must send a control message $\pi$ to Process~C using already
% established link. It uses Process~A as mediator and buffers its delivered
% messages while awaiting for Process~C's reply. In
% Figure~\ref{fig:spaceproblemC}, Process~A broadcasts $a$ along with its assigned
% unique identifier $\langle A,\, 1 \rangle$. Process~A updates its local vector
% accordingly. Then, Process~A routes $\pi$ to Process~C. Process~B receives $b$
% but discards it, for its local vector already contains this message.  Process~C
% receives, saves, delivers, and forwards $b$. In Figure~\ref{fig:spaceproblemD},
% Process~A receives and discards $b$. Process~B receives, saves, delivers,
% buffers, and forwards $a$. Process~C receives, saves, delivers, and forwards
% $a$. In Figure~\ref{fig:spaceproblemE}, Process~A receives two copies of an
% already delivered message $a$, hence discards them both. Process~C receives
% $\pi$ and sends its reply $\rho$ to Process~B. Such reply can travel using any
% communication mean. In Figure~\ref{fig:spaceproblemF}, Process~B receives
% $\rho$. Process~B empties its buffer comprising $a$ to Process~C. This procedure
% ensures that the delivery of $b$ happens before the delivery of $a$ at
% Process~C. Afterwards, Process~B uses this link normally for causal broadcast.

% This scenario highlights the issue about memory consumption: the local structure
% to discards multiple receipts increases linearly with the number of processes
% that ever broadcast a message.

\begin{figure*}
  \begin{center}
    \begin{subfigure}[t]{0.3\textwidth}
      \input{input/figcounterproblemA.tex}
      \caption{\label{fig:counterproblemA}Process~B broadcasts the 
        message $b$ and expects to receive 1 copy of $b$.}
    \end{subfigure}
    \begin{subfigure}[t]{0.3\textwidth}
      \input{input/figcounterproblemB.tex}
      \caption{\label{fig:counterproblemB}Process~A receives, delivers
        and forwards $b$. It expects 1 additional copy of $b$. Process~B wants to 
        add Process~C in its direct neighbors for causal broadcast. It sends a 
        control message $\pi$ to Process~C using Process~A as mediator.}
    \end{subfigure}
    \begin{subfigure}[t]{0.3\textwidth}
      \input{input/figcounterproblemC.tex}
      \caption{\label{fig:counterproblemC}Process~A broadcasts $a$. It 
        expects 2 copies. Process~B receives $b$ again, it does not expect any other 
        copy. Hence, it purges its local structure from $b$. Process~C receives, 
        delivers and forwards $b$. It does not expect additional copies.}
    \end{subfigure}
    \begin{subfigure}[t]{0.3\textwidth}
      \input{input/figcounterproblemD.tex}
      \caption{\label{fig:counterproblemD}Process~A receives, discards,
        and purges $b$. Process~B and Process~C receive, deliver, and forward $a$ but
        do not expect any additional copy. Process~B buffers $a$ to ensure the safety 
        of the new link.}
    \end{subfigure}
    \begin{subfigure}[t]{0.3\textwidth}
      \input{input/figcounterproblemE.tex}
      \caption{\label{fig:counterproblemE}Process~A receives and discards 
        two copies of $a$. It purges its local structure of $a$. Process~C receives
        $\pi$ and replies $\rho$ to Process~B.}
    \end{subfigure}
    \begin{subfigure}[t]{0.3\textwidth}
      \input{input/figcounterproblemF.tex}
      \caption{\label{fig:counterproblemF}Process~B empties its buffer to
        Process~C. Not only Process~C mistakes $a$ for a new message and
        delivers it again, but it has cascading effects due to the forwarding.}
    \end{subfigure}
    \caption{\label{fig:counterproblem}Using counters to discard multiple
      receipts is efficient in terms of memory usage but fails in dynamic
      systems.}
  \end{center}
\end{figure*}

% However, we observe that information about messages becomes obsolete over
% receipts. For instance, all copies of the message $b$ disappear of the system in
% Figure~\ref{fig:spaceproblemD}. With such knowledge, processes could remove this
% message from their local structure safely. 

To remove this last linear upper bound, Figure~\ref{fig:counterproblem} hints
that, instead of a costly data structure summarizing the global state of the
system, a lightweight data structure that is purged over receipts can ensure
exactly once delivery as long as the system remains static. In this scenario,
processes use \PCBROADCAST. When a process broadcasts a message, it sends it to
its set of neighbors (see Figures~\ref{fig:counterproblemA}
and~\ref{fig:counterproblemC}); when a process delivers such message, it
forwards it to its own neighbors in a gossip fashion (see
Figure~\ref{fig:counterproblemB}). Processes receive each message either
directly or transitively.\\
In addition, to ensure exactly once delivery, each process maps each delivered
message with a number of expected copies. Each processes computes this number of
expected copies depending on its number of incoming links. For instance, in
Figure~\ref{fig:counterproblemA}, Process~B broadcasts $b$ and expects 1 copy of
this message, for Process~A will deliver and forward $b$ exactly once. The
receipt of any copy decrements its corresponding entry.  When the number of
expected copies reaches 0, the process removes the entry altogether (see
Figures~\ref{fig:counterproblemC} and~\ref{fig:counterproblemE}).\\ We observe
that counters stay consistent until Process~B adds Process~C in its
neighborhood. \PCBROADCAST prevents causal order violations by the mean of a
round-trip of control messages.  At first, Process~B cannot communicate directly
with Process~C for causal broadcast. When Process~B wants to add Process~C in
its neighborhood, it must send a control message $\pi$ using its established
links to Process~C (see
Figures~\ref{fig:counterproblemB}--\ref{fig:counterproblemD}, await for the
latter's answer (see Figure~\ref{fig:counterproblemE}) while buffering each
message it delivers (see Figure~\ref{fig:counterproblemD}). Upon receipt of the
acknowledgment $\rho$, Process~B empties its buffer to Process~C and starts
using the link normally for causal broadcast. However,
Figure~\ref{fig:counterproblemF} shows that not only this leads to the double
delivery of $a$ but also this has cascading effects. Process~C was unable to
compute the number of expected messages correctly. Process~C was unable to
distinguish messages it would receive from Process~B (i.e., $a$) from messages
it would never receive from Process~B (i.e., $b$).

% depicts the same scenario where we replace the linearly increasing local
% structure by a set of messages purged over receipts. Assuming that each process
% delivers and forwards each message exactly once, processes expect to receive one
% copy of each broadcast message per link pointing to them. For instance, in
% Figure~\ref{fig:counterproblemA}, Process~B expects to receive one copy of $b$,
% for only Process~A has Process~B as neighbor; in
% Figure~\ref{fig:counterproblemC}, Process~A expects two copies of $a$, for both
% Process~B and Process~C have Process~A as neighbor. Received messages that are
% not in the local structure are new and must be delivered. Receives messages that
% are in the local structure must be ignored and the counter of expected copies
% decremented. When this counter reaches 0, the process assumes that it will not
% receive this message again, hence it purges its local structure from this
% message. For instance, in Figure~\ref{fig:counterproblemC}, Process~B removes
% the element $b$, for it received the awaited copy; in
% Figure~\ref{fig:counterproblemE}, Process~A removes the element $a$, for it
% received both awaited copies.  Unfortunately, in
% Figure~\ref{fig:counterproblemF}, the number of expected messages becomes
% inconsistent due to dynamicity. Process~B empties its buffer of messages
% containing $a$, for the sake of causal order. When Process~C receives $a$, it
% mistakes this message for a new message. Not only Process~C delivers $a$ a
% second time, but it forwards it to its neighbors leading to a cascading effect
% of multiple deliveries.


% Even if counters are insufficient to discard multiple receipts in dynamic
% systems, this hints that broadcast can guarantee reliability locally without
% maintaining costly data structures summarizing the global state of the system,
% e.g., using vector clocks.


% To solve this issue, state-of-the-art protocols maintain a local vector the size
% of which increases linearly with the number of processes that ever broadcast a
% message. They eventually become overcostly in dynamic settings.


In this paper, we exploit and extend \PCBROADCAST to provide a causal broadcast
middleware that is lightweight in terms of local memory consumption, and message
overhead. Table~\ref{table:complexity} shows the complexity of the proposed
approach. It keeps constant message overhead and constant delivery execution
time while removing the last linear dependency $N$ that remained on local space
complexity. Its overhead in terms of number of control messages is twice that of
\PCBROADCAST.
%% It maintains a local structure that grows and shrinks over time when
%% processes can join, leave, or self-reconfigure their neighborhood over time.
%% \TODO{maybe more details.}
The next section describes the proposed protocol.


%%% Local Variables:
%%% mode: latex
%%% TeX-master: "../paper"
%%% End:
