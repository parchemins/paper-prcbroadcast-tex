
\section{Conclusion}
\label{sec:conclusion}


In this paper, we proposed an original causal broadcast protocol that removes
the last linear upper bound in terms of number of processes that ever broadcast
a message that remained on local space complexity. This approach only uses
buffers of messages that grow and shrink when processes add new neighbors. They
eventually become empty when the system becomes static. To achieve this, our
approach only uses small control messages of constant size. The number of
control messages depends on the overlay network. Using routing strategies, this
number remains small, hence generated traffic overhead remains small. This
advantageous trade-off makes causal broadcast a lightweight and efficient
middleware for group communication in distributed systems. This advantageous
trade-off even makes \RPCBROADCAST a lightweight and efficient implementation
for reliable broadcast. As consequence, causal broadcast and reliable broadcast
can run in large and dynamic systems even on most humble devices such as
Raspberry Pi’s.




%%% Local Variables:
%%% mode: latex
%%% TeX-master: "../paper"
%%% End:
