
%\IEEEtitleabstractindextext{
\begin{abstract}
  Causal broadcast constitutes the fundamental communication primitive of many
  distributed protocols and applications. However, state-of-the-art
  implementations fail to forget obsolete control information about past
  delivered messages. They do not scale in large and dynamic systems.  In this
  paper, we propose a novel implementation of causal broadcast.  We prove that
  all and only obsolete control information is safely removed, at cost of a few
  lightweight control messages. Local space complexity is not monotonically
  increasing and depends on system settings and use.  Message overhead is
  constant.  Our implementation constitutes a sustainable communication
  primitive for causal broadcast in large and dynamic systems.
\end{abstract}
%}

% \begin{abstract}
%   Causal broadcast constitutes the core communication primitive of many
%   distributed systems. For decades, state-of-the-art approaches relied on
%   maintaining and transmitting vector clocks. The size of vector clocks
%   increases linearly with the number of processes that ever entered the
%   system. Causal broadcast eventually became overcostly and inefficient in large
%   and dynamic systems.  A recent approach solved the issue about generated
%   traffic by removing the need for transmitting vectors. However, it still
%   maintains a vector locally. In this paper, we improve this causal broadcast by
%   removing the need for such vector. The proposed protocol safely purges the
%   local structure over time at cost of few control messages. As consequence,
%   causal broadcast can run in large and dynamic systems even on most humble
%   devices such as Raspberry Pi's.
% \end{abstract}

% \keywords{Causal broadcast, local space complexity}


%%% Local Variables:
%%% mode: latex
%%% TeX-master: "../paper"
%%% End:
