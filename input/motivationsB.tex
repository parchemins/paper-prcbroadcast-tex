
\section{Issues and motivations}
\label{sec:motivations}

A distributed system is set of processes that can communicate with each other
using messages. Processes may only have partial knowledge of the full system
membership, for maintaining full knowledge is overcostly in large and dynamic
systems where millions of processes could join, leave, or self-reconfigure at
any time~\cite{birman1999bimodal,demers1987epidemic}. Broadcasting is a
communication primitive that allows a process to send a message to all processes
that belong to its system. Reliable broadcast ensures that all messages reach
all processes and processes deliver each message exactly once. Causal broadcast
is a reliable broadcast that ensures causal order on message
delivery~\cite{hadzilacos1994modular}:
% Causal broadcast is a communication primitive that allows a process to send
% messages to all processes of its distributed system (\REF). Message deliveries
% follow the happen before relationship (\REF).
if the sending of a message $m$ precedes the sending of a message $m'$ then all
processes that deliver these two messages need to deliver $m$ before $m'$;
otherwise they deliver them in any order.

Implementing reliable broadcast requires vectors of Lamport's
clocks~\cite{lamport1978time} to forbid multiple delivery of broadcast
messages. However, the size of such vectors monotonically increases with the
number of processes that ever broadcast a message. The cost becomes
prohibitively high in contexts where anyone can broadcast a message at any
time~\cite{borthakur2013petabyte,nedelec2016crate,slate13}. One million
broadcasters implies that each process maintains locally a vector with one
million entries. Processes cannot safely reclaim entries without running and
overcostly consensus protocol.

\begin{figure*}
  \begin{center}
    \begin{subfigure}[t]{0.31\textwidth}
      
\begin{tikzpicture}[scale=1]
  
  \small
  
  \newcommand\X{210/5pt};
  \newcommand\Y{30pt};

  
  \draw[fill=white] (0*\X, 0*\Y) node{\textbf{A}} +(-5pt, -5pt) rectangle +(5pt, 5pt);
  \draw (-5+0*\X, 5+0*\Y) node[above]{$\varnothing$};
  \draw[fill=white] (1*\X, -1*\Y) node{\textbf{B}} +(-5pt, -5pt) rectangle +(5pt, 5pt);
  \draw (1*\X, -5-1*\Y) node[below]{$\{\bm{b:1}\}$};
  \draw[fill=white] (2*\X,  0*\Y) node{\textbf{C}} +(-5pt, -5pt) rectangle +(5pt, 5pt);
  \draw (5+2*\X, 5+0*\Y) node[above]{$\varnothing$};
  
  \draw[->](5+0*\X, 0*\Y) -- (-5+1*\X, -1*\Y); %% A->B
  \draw[<-](5+0*\X, -5+0*\Y) --
  node[sloped,below]{$\bm{b}$}
  (-5+1*\X, -5-1*\Y); %% A<-B
  
  \draw[->](5+0*\X, 5+0*\Y) -- (-5+2*\X, 5+0*\Y); % A->C
  \draw[<-](5+0*\X,  1.25+ 0*\Y) -- (-5+2*\X,  1.25+ 0*\Y); % A<-C
  
  % \draw[->,dashed](5+1*\X, -1*\Y) -- (-5+2*\X, 0*\Y); %% B<-C
  % \draw[->, dashed](5+1*\X, -5-1*\Y) -- (-5+2*\X, -5+0*\Y); %% B->C



\end{tikzpicture}%
      \caption{\label{fig:counterproblemA}Process~B broadcasts the 
        message $b$ and expects to receive 1 copy of $b$.}
    \end{subfigure}
    ~
    \begin{subfigure}[t]{0.31\textwidth}
      
\begin{tikzpicture}[scale=1]
  
  \small
  
  \newcommand\X{210/5pt};
  \newcommand\Y{30pt};

  
  \draw[fill=white] (0*\X, 0*\Y) node{\textbf{A}} +(-5pt, -5pt) rectangle +(5pt, 5pt);
  \draw (-5+0*\X, 5+0*\Y) node[above]{$\{\bm{b:1}\}$};
  \draw[fill=white] (1*\X, -1*\Y) node{\textbf{B}} +(-5pt, -5pt) rectangle +(5pt, 5pt);
  \draw (1*\X, -5-1*\Y) node[below]{$\{b:1\}$};
  \draw[fill=white] (2*\X,  0*\Y) node{\textbf{C}} +(-5pt, -5pt) rectangle +(5pt, 5pt);
  \draw (5+2*\X, 5+0*\Y) node[above]{$\varnothing$};
  
  \draw[->](5+0*\X, 0*\Y) -- 
  node[sloped,above]{$b$}
  (-5+1*\X, -1*\Y); %% A->B

  \draw[<-](5+0*\X, -5+0*\Y) --
  node[sloped, below]{$\bm{\pi}$}
  (-5+1*\X, -5-1*\Y); %% A<-B
  
  \draw[->](5+0*\X, 5+0*\Y) --
  node[sloped,above]{$b$}
  (-5+2*\X, 5+0*\Y); % A->C
  
  \draw[<-](5+0*\X,  1.25+ 0*\Y) -- (-5+2*\X,  1.25+ 0*\Y); % A<-C
  
  % \draw[->,dashed](5+1*\X, -1*\Y) -- (-5+2*\X, 0*\Y); %% B<-C
  \draw[->, dashed](5+1*\X, -5-1*\Y) -- (-5+2*\X, -5+0*\Y); %% B->C



\end{tikzpicture}%
      \caption{\label{fig:counterproblemB}Process~A receives, delivers
        and forwards $b$. It expects 1 additional copy of $b$. Process~B wants to 
        add Process~C in its direct neighbors for causal broadcast. It sends a 
        control message $\pi$ to Process~C using Process~A as mediator.}
    \end{subfigure}
    ~
    \begin{subfigure}[t]{0.31\textwidth}
      
\begin{tikzpicture}[scale=1]
  
  \small
  
  \newcommand\X{210/5pt};
  \newcommand\Y{30pt};

  
  \draw[fill=white] (0*\X, 0*\Y) node{\textbf{A}} +(-5pt, -5pt) rectangle +(5pt, 5pt);
  \draw (-5+0*\X, 5+0*\Y) node[above]{$\{\bm{a:2},\,b:1\}$};
  \draw[fill=white] (1*\X, -1*\Y) node{\textbf{B}} +(-5pt, -5pt) rectangle +(5pt, 5pt);
  \draw (1*\X, -5-1*\Y) node[below]{$\bm{\varnothing}$\vphantom{$\{$}};
  \draw[fill=white] (2*\X,  0*\Y) node{\textbf{C}} +(-5pt, -5pt) rectangle +(5pt, 5pt);
  \draw (5+2*\X, 5+0*\Y) node[above]{$\varnothing$};
  
  \draw[->](5+0*\X, 0*\Y) -- 
  node[sloped,above]{$\bm{a}$}
  (-5+1*\X, -1*\Y); %% A->B

  \draw[<-](5+0*\X, -5+0*\Y) --
  (-5+1*\X, -5-1*\Y); %% A<-B
  
  \draw[->](5+0*\X, 5+0*\Y) --
  node[above]{~~$\pi$~~~$\bm{a}$}
  (-5+2*\X, 5+0*\Y); % A->C
  
  \draw[<-](5+0*\X,  1.25+ 0*\Y) --
  node[below]{ ~ ~ ~ ~ ~$b$}
 (-5+2*\X,  1.25+ 0*\Y); % A<-C
  
  % \draw[->,dashed](5+1*\X, -1*\Y) -- (-5+2*\X, 0*\Y); %% B<-C
  \draw[->, dashed](5+1*\X, -5-1*\Y) -- (-5+2*\X, -5+0*\Y); %% B->C



\end{tikzpicture}%
      \caption{\label{fig:counterproblemC}Process~A broadcasts $a$. It 
        expects 2 copies. Process~B receives $b$ again, it does not expect any other 
        copy. Hence, it purges its local structure from $b$. Process~C receives, 
        delivers and forwards $b$. It does not expect additional copies.}
    \end{subfigure}
    \begin{subfigure}[t]{0.31\textwidth}
      
\begin{tikzpicture}[scale=1]
  
  \small
  
  \newcommand\X{210/5pt};
  \newcommand\Y{30pt};

  
  \draw[fill=white] (0*\X, 0*\Y) node{\textbf{A}} +(-5pt, -5pt) rectangle +(5pt, 5pt);
  \draw (-5+0*\X, 5+0*\Y) node[above]{$\{a:2\}$};
  \draw[fill=white] (1*\X, -1*\Y) node{\textbf{B}} +(-5pt, -5pt) rectangle +(5pt, 5pt);
  \draw (1*\X, -5-1*\Y) node[below]{$\varnothing$\vphantom{$\{$}};
  \draw[fill=white] (2*\X,  0*\Y) node{\textbf{C}} +(-5pt, -5pt) rectangle +(5pt, 5pt);
  \draw (5+2*\X, 5+0*\Y) node[above]{$\varnothing$};
  
  \draw[->](5+0*\X, 0*\Y) -- 
  (-5+1*\X, -1*\Y); %% A->B

  \draw[<-](5+0*\X, -5+0*\Y) --
  node[sloped,below]{$a$}
  (-5+1*\X, -5-1*\Y); %% A<-B
  
  \draw[->](5+0*\X, 5+0*\Y) --
  node[above]{$\pi$}
  (-5+2*\X, 5+0*\Y); % A->C
  
  \draw[<-](5+0*\X,  1.25+ 0*\Y) --
  node[below]{$a$}
  (-5+2*\X,  1.25+ 0*\Y); % A<-C
  
  % \draw[->,dashed](5+1*\X, -1*\Y) -- (-5+2*\X, 0*\Y); %% B<-C
  \draw[->, dashed](5+1*\X, -5-1*\Y) --
  node[sloped, below]{$[a]$} (-5+2*\X, -5+0*\Y); %% B->C



\end{tikzpicture}%
      \caption{\label{fig:counterproblemD}Process~A receives, discards,
        and purges $b$. Process~B and Process~C receive, deliver, and forward $a$ but
        do not expect any additional copy. Process~B buffers $a$ to ensure the safety 
        of the new link.}
    \end{subfigure}
    ~
    \begin{subfigure}[t]{0.31\textwidth}
      
\begin{tikzpicture}[scale=1]
  
  \small
  
  \newcommand\X{210/5pt};
  \newcommand\Y{30pt};

  
  \draw[fill=white] (0*\X, 0*\Y) node{\textbf{A}} +(-5pt, -5pt) rectangle +(5pt, 5pt);
  \draw (-5+0*\X, 5+0*\Y) node[above]{$\varnothing$};
  \draw[fill=white] (1*\X, -1*\Y) node{\textbf{B}} +(-5pt, -5pt) rectangle +(5pt, 5pt);
  \draw (1*\X, -5-1*\Y) node[below]{$\varnothing$\vphantom{$\{$}};
  \draw[fill=white] (2*\X,  0*\Y) node{\textbf{C}} +(-5pt, -5pt) rectangle +(5pt, 5pt);
  \draw (5+2*\X, 5+0*\Y) node[above]{$\varnothing$};
  
  \draw[->](5+0*\X, 0*\Y) -- 
  (-5+1*\X, -1*\Y); %% A->B

  \draw[<-](5+0*\X, -5+0*\Y) --
  (-5+1*\X, -5-1*\Y); %% A<-B
  
  \draw[->](5+0*\X, 5+0*\Y) --
  (-5+2*\X, 5+0*\Y); % A->C
  
  \draw[<-](5+0*\X,  1.25+ 0*\Y) --
  (-5+2*\X,  1.25+ 0*\Y); % A<-C
  
  % \draw[->,dashed](5+1*\X, -1*\Y) -- (-5+2*\X, 0*\Y); %% B<-C
 \draw[<-,densely dotted,decorate, decoration={snake, amplitude=0.3mm}](5+5+1*\X, 5+-1*\Y) -- 
 node[above, sloped]{$\bm{\rho}$}
 (-5+2*\X, 0*\Y);

  \draw[->, dashed](5+1*\X, -5-1*\Y) --
  node[sloped, below]{$[a]$} (-5+2*\X, -5+0*\Y); %% B->C



\end{tikzpicture}%
      \caption{\label{fig:counterproblemE}Process~A receives and discards 
        two copies of $a$. It purges its local structure of $a$. Process~C receives
        $\pi$ and replies $\rho$ to Process~B.}
    \end{subfigure}
    ~
    \begin{subfigure}[t]{0.31\textwidth}
      
\begin{tikzpicture}[scale=1]
  
  \small
  
  \newcommand\X{210/5pt};
  \newcommand\Y{30pt};

  
  \draw[fill=white] (0*\X, 0*\Y) node{\textbf{A}} +(-5pt, -5pt) rectangle +(5pt, 5pt);
  \draw (-5+0*\X, 5+0*\Y) node[above]{$\varnothing$};
  \draw[fill=white] (1*\X, -1*\Y) node{\textbf{B}} +(-5pt, -5pt) rectangle +(5pt, 5pt);
  \draw (1*\X, -5-1*\Y) node[below]{$\varnothing$\vphantom{$\{$}};
  \draw[fill=white] (2*\X,  0*\Y) node{\textbf{C}} +(-5pt, -5pt) rectangle +(5pt, 5pt);
  \draw (5+2*\X, 5+0*\Y) node[above]{$\{\bm{a:1}\}$};
  
  \draw[->](5+0*\X, 0*\Y) -- 
  (-5+1*\X, -1*\Y); %% A->B

  \draw[<-](5+0*\X, -5+0*\Y) --
  (-5+1*\X, -5-1*\Y); %% A<-B
  
  \draw[->](5+0*\X, 5+0*\Y) --
  (-5+2*\X, 5+0*\Y); % A->C
  
  \draw[<-](5+0*\X,  1.25+ 0*\Y) --
  node[sloped, below]{ ~ ~ ~ ~ ~ ~ ~ ~~$\bm{a}$}
  (-5+2*\X,  1.25+ 0*\Y); % A<-C
  
  % \draw[->,dashed](5+1*\X, -1*\Y) -- (-5+2*\X, 0*\Y); %% B<-C
  \draw[->](5+1*\X, -5-1*\Y) --
  (-5+2*\X, -5+0*\Y); %% B->C



\end{tikzpicture}%
      \caption{\label{fig:counterproblemF}Process~B empties its buffer to
        Process~C. Not only Process~C mistakes $a$ for a new message and
        delivers it again, but it has cascading effects due to the forwarding.}
    \end{subfigure}
    \caption{\label{fig:counterproblem}Using counters to forbid multiple
      delivery is lightweight but fails in dynamic systems.}
  \end{center}
\end{figure*}


To remove this last monotonic and linear upper bound on space complexity,
Figure~\ref{fig:counterproblem} shows that summarizing the global state of the
system using vector clocks is unnecessary. Instead, a simple and lightweight
data structure can forbid multiple delivery as long as the system remains
static~\cite{raynal2013distributed}. In this scenario, processes use
\PCBROADCAST (\REF): a causal broadcast protocol that ensures causal order of
messages even in dynamic systems without message overhead.  When a process
broadcasts a message, it sends it to its set of neighbors (see
Figures~\ref{fig:counterproblemA} and~\ref{fig:counterproblemC}); when a process
delivers such message, it forwards it to its own neighbors in a gossip fashion
(see Figure~\ref{fig:counterproblemB}). Processes receive each message either
directly or transitively.\\
In addition, to ensure exactly once delivery, each process maps each delivered
message with a number of expected copies. Each processes computes this number of
expected copies depending on its number of incoming links. For instance, in
Figure~\ref{fig:counterproblemA}, Process~B broadcasts $b$ and expects 1 copy of
this message, for Process~A will deliver and forward $b$ exactly once. The
receipt of any copy decrements its corresponding entry.  When the number of
expected copies reaches 0, the process removes the entry altogether (see
Figures~\ref{fig:counterproblemC} and~\ref{fig:counterproblemE}).\\ We observe
that counters stay consistent until Process~B adds Process~C in its
neighborhood. \PCBROADCAST prevents causal order violations by the mean of a
round-trip of control messages.  At first, Process~B cannot communicate directly
with Process~C for causal broadcast. When Process~B wants to add Process~C in
its neighborhood, it must send a control message $\pi$ using its established
links to Process~C (see
Figures~\ref{fig:counterproblemB}--\ref{fig:counterproblemD}), await for the
latter's answer (see Figure~\ref{fig:counterproblemE}) while buffering each
message it delivers (see Figure~\ref{fig:counterproblemD}). Upon receipt of the
acknowledgment $\rho$, Process~B empties its buffer to Process~C and starts
using the link normally for causal broadcast. However,
Figure~\ref{fig:counterproblemF} shows that not only this leads to the double
delivery of $a$ but also this has cascading effects. Process~C was unable to
compute the number of expected messages correctly. Process~C was unable to
distinguish messages it would receive from Process~B (i.e., $a$) from messages
it would never receive from Process~B (i.e., $b$).

In this paper, we exploit and extend \PCBROADCAST to provide a causal broadcast
middleware that keeps constant message overhead while removing the last linear
and monotonic dependency $O(N)$ that remained on local space complexity (see
Table~\ref{table:complexity}). Compared to \PCBROADCAST, its overhead in terms
of number of control messages is twice higher.  The next section describes the
proposed protocol.


%%% Local Variables:
%%% mode: latex
%%% TeX-master: "../paper"
%%% End:
